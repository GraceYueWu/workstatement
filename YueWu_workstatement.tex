\documentclass[12pt,letterpaper]{article}

\usepackage{amsmath, amsthm, amssymb, amsfonts}
\usepackage{graphicx}
\usepackage{bm}
\usepackage{natbib}

\theoremstyle{definition}
\newtheorem{dfn}{Definition}

\begin{document}

% The numbers below controls the amount of space between the following sections
\def\shiftdowna{0.32in}  % Adjust for balance
\def\shiftdownb{0.22in}  % Adjust for balance

% Set up the boiler plate at the top of the page

\begin{center}
\textbf{{\large Project Work Statement}}\\


% SPONSOR
\vspace \shiftdowna
\underline {Sponsor}\\ 
\vspace{5pt}
\textbf{{\large BGE (Baltimore Gas and Electric)}}\\


% TITLE
\vspace \shiftdowna
\textbf{{\large Effect from imperfect information in different electricity market}}


% STUDENTS
\vspace{0.35in}
\vspace \shiftdownb
\underline {Participants} \\
\vspace{5pt}
\text{Yue Wu}, \texttt{ywu67@jhu.edu}


% DATE
\vspace \shiftdowna
Date: \today

\end{center}

\vfill  
%Fill page to force following note to bottom
\footnoterule
\noindent \small{Any apparent association of this work to BGE is
fictional one, and the sole purpose of this work is a class exercise}

\newpage

\section{Background} 
BGE is a subsidiary of Exelon Corporation, the nation's leading competitive energy provider with approximately 33 billion dollars in annual revenues. For nearly 200 years, BGE has been the innovator in meeting the energy needs of Central Maryland residents and businesses with a growing array of programs, services and resources that are defining the next generation of energy management. This innovation continues today as we serve more than 1.2 million electric customers and more than 650,000 gas customers in an economically diverse, 2,300-square-mile area encompassing Baltimore City and all or part of 10 Central Maryland counties.

\section{Problem Statement}
In the electricity market, companies will have to make decision everyday. For example, in a monopoly market, companies will have to make operation strategy; which including how many assets to use, what kind of fuel to use, how many capacity to reserve, etc. By making these decisions, the utility wants to maximize their profit or minimize their cost while fufill market requiremet by a particualr demand on a certain price. In another kind of typical market, which is competitive market, electricity companies will have to make strategy to compete with each other. For example, in a one-day ahead bidding market, companies bid to produce for the next day. They will bid for the price they are going to sell the electricity as well as amount of production.  After the biding, some of they get the right to produce the particular amount they bid at a certain price (may or may not be their bidding price) the next day. To plan a bidding strategy, the company also wants to maximize the profit or minimize cost. 
All the above decisions will have to be made according to their information of what demand and price may be the next day. How much a better prediction of load and price will affect company profit and behavior is what the project is going to find out.

\section{Approach}
Given the limited time, we will assume a linear optimization model for the market. Demand curve of the market will be linear; prediction error will be assumed normally distributed with known mean value and standard deviation. In each scenario, companies try to maximize their profit or minimize their cost based on their information. Two typical market scenarios are going to be studied: 
\begin{itemize}
\item Monopoly market: where only one company is in the market with market price fixed. The company make operation decision to minimize their cost while fulfilling electricity demand.
\item Completetly competitive market: where their are many companies in the market. The market power of each company is so small that no single company can actually affect market price and demand. All the companies will bid for the right to produce for the next day. For a single company, they will have to find out their best bid to maximize their profit based on prediction.
\end{itemize}
At last, compare the model output under different information accuracy, such as operation strategy and expected cost/profit.

     
\section{Milestones}
We have the following major deadlines:
\begin{itemize}
    \item Work Statement due date, Sep 28, 2012,
    \item Midterm Presentation due date, Oct 12, 2012,
    \item Progress Report due date, Oct 26, 2012,
    \item Final Presentation due date, Nov 6, 2012,
    \item Final Report due date, Nov 30, 2012.
\end{itemize}

\section{Deliverable}
\subsection{From Team to Sponsor} % (fold)
The following outputs are expected from this project:
\begin{itemize}
    \item List of economics results under different scenarios,
    \item Mathematical description of market models, including data and equations,
    \item Spreadsheet of Excel showing details of how the model is structured and solved,
    \item Technical report and presentations summarizing the work.  
\end{itemize}
\subsection{From Sponsor to Team} % (fold)

In order for our project to be of successful one, we will need:
\begin{itemize}
    \item Data of electricity demand and price,
    \item Current predicting accuracy and expected predicting accuracy in the future.

\end{itemize}
We expect all the data above delivered before Oct 15, 2012. State wide data from will be adjusted and used as replacement in case of data being unavailable. 


\bibliographystyle{plain}
\renewcommand\bibname{Selected Bibliography Including Cited Works}
\nocite{*}
\bibliography{biblio}

\end{document}
